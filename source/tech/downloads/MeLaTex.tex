
\documentclass[a4paper]{report}
\usepackage{indentfirst}
\usepackage{amsmath}
\usepackage{amssymb}
\usepackage{mathrsfs}
% Package eufrak Warning: The eufrak package is redundant if the amsfonts package is used on input line 36.
% \usepackage{eufrak}

% http://blog.jqian.net/post/xelatex.html
\usepackage{xltxtra,fontspec,xunicode}
\usepackage[slantfont,boldfont]{xeCJK} % 允许斜体和粗体
% http://mirror.lzu.edu.cn/CTAN/macros/xetex/latex/xecjk/xeCJK.pdf
% Arial Unicode MS
% FangSong,仿宋
% KaiTi,楷体
% Microsoft YaHei,微软雅黑
% MingLiU,細明體
% NSimSun,新宋体
% PMingLiU,新細明體
% SimHei,黑体
% SimSun,宋体  
\setCJKmainfont{FangSong}          % 设置缺省中文字体
\setCJKmonofont{Microsoft YaHei}   % 设置等宽字体
\setCJKsansfont{Microsoft YaHei}   % 设置无衬线字体
% \setmainfont{Arial}   % 英文衬线字体
% \setmonofont{Consolas}   % 英文等宽字体
% \setsansfont{Calibri} % 英文无衬线字体
\setmainfont{Arial}   % 英文衬线字体
\setmonofont{Microsoft YaHei}   % 英文等宽字体
\setsansfont{Calibri} % 英文无衬线字体

% https://en.wikibooks.org/wiki/LaTeX/Hyperlinks
\usepackage{hyperref}

\newtheorem{Theorem}{定理}[section]
\newtheorem{FirstTheorem}{定理}
\newtheorem{NextTheorem}[FirstTheorem]{定理}
\title{Me Into \LaTeX}
\author{Utensil\thanks{Thanks to the reader.}}
\date{\today}
\begin{document}
\maketitle
\chapter*{Preface\footnote{This chapter will not be enumerated.}}

终于将这个我的第一个\TeX{}文档写完了。这份文档简单地介绍了\TeX{}的基本知识、常用环境,以及数学环境。\\

写这个文档的初衷其实是练习,这使得这个文档的源代码是和它讲述的\TeX{}知识共同成长的,我学到了什么程度,就用
什么程度的\TeX{}命令来进行排版。甚至在许多地方,文档和其源代码是密不可分的整体,共同完成对\TeX{}知识的介绍。
例如,在一开始的许多地方,我没有在文档中同时展示源代码和其效果,我只在文档中给出其效果,而让读者自行查找是
什么源代码实现了这个效果$\cdots\cdots$\\

这份文档虽然全部由英文写成,不过这里头很多陌生单词是和命令名一样的,其余则是生词不超过20个
的浅易英文。\\

这份文档现在是在Visual Studio Code下编辑,并直接使用Tex Live 2017自带的xelatex编译的。(这份文档曾经是在C\TeX{}下完成和进行编译的,感谢为C\TeX{}付出的所有人!)\\

大家有什么问题可以在我的博客 http://utensil.github.io/ 留言,或者发邮件到 Utensil Candel At Gmail 。\\

\tableofcontents
\chapter{Basics}
\section{Spaces and Reserved Symbols}

It does not matter whether you enter one or several spaces after a
word.

An empty line starts a new paragraph.

These symbols have to be slashed: \# \$ \% \^{} \& \_ \{ \} \~{}

But if we slash $\backslash$ will get an \\ line break,it's the same
as the
\newline $\backslash$newline.

\LaTeX{} will ignore the spaces after an order.

\ldots

\section{Hyphenation}

We can tell \LaTeX\ how to hyphenate,for example,this long long word: su\-per\-cal\-%
i\-frag\-i\-lis\-tic\-ex\-pi\-%
al\-i\-do\-cious.

We can tell \LaTeX\ not to hyphenate,for example,this long long
word: \mbox{supercalifragilisticexpialidocious}.

This will cause an ``underfull hbox''.

If we lower the quality demand,\LaTeX\ will do it like this:\sloppy
\mbox{supercalifragilisticexpialidocious}.

That's horrible,isn't it?So we have to resume it.\fussy

We can draw a quad around the
texts:\fbox{supercalifragilisticexpialidocious}

\section{Special Symbols}

``sth''

`another sth'

-

--

---

$-1$

sth\~ sth

sth\~{}sth

$\sim$

$-30\,^{\circ}\mathrm{C}$

ff

f\mbox{}f

H\^otel,

na\"\i ve

\'el\`eve

sm\o rrebr\o d

!`Se\~norita!

Sch\"onbrunner Schlo\ss{} Stra\ss e

\`o{} \'o{} \^o{} \~o{}  \=o{}  \.o{} \"o{}


\c c \u o \v o \H o \c o  \d o \b o \t oo

\oe{} \OE{} \ae{} \AE{} \aa{} \AA{}

\o{} \O{} \l{} \L{} \i{} \j{}
!`{} ?`



Mr.~Smith was happy to see her.

I like BASIC\@. What about you?

\section{Structrue}

$\backslash$documentclass[options]\{class\}

\subsection{Classes}

article

report

book

slides

\subsection{Options}

10pt[ 11pt,12pt\ldots]

letterpaper [
a4paper,a5paper,b5paper,executivepaper,legalpaper\ldots]

fleqn: Left align the math formulas.

leqno: Put the serial number of math formulas on its left.

titlepage, notitlepage

onecolumn, twocolumn

twoside, oneside

openright, openany: Where the new chapter starts.

\subsection{Layers}


$\backslash$part\{\ldots\}

$\backslash$chapter\{\ldots\}

$\backslash$section\{\ldots\}

$\backslash$subsection\{\ldots\}

$\backslash$subsubsection\{\ldots\}

$\backslash$paragragh\{\ldots\}

$\backslash$subparagragh\{\ldots\}

\section{Cross Reference}
\label{Cross Reference} See section \ref{Cross Reference} on page
\pageref{Cross Reference}.

\section{FootNote}
See the footnote\footnote{We don't need to say anything here.}.

\section{Emphasizing}
You can use \underline{$\backslash$underline},but
\emph{$\backslash$emph} is recommended.

\textit{You can also \emph{emphasize} text if it is set in italics,}
\textsf{in a \emph{sans-serif} font,} \texttt{or in
\emph{typewriter} style.}

\section{Text Fonts}
\textrm{Roman}

\textsf{Sans Serif}

\texttt{Typewriter}

\textmd{medium}

\textbf{Bold Face}

\textup{Upright}

\textit{italic}

\textsl{slanted}

\textsc{Small Caps}

\section{Text Size}

\tiny{Tiny}

\scriptsize{Scriptsize}

\footnotesize{Footnotesize}

\small{Small}

\normalsize{Normalsize}

\large{large}

\Large{Large}

\LARGE{LARGE}

\huge{huge}

\Huge{Huge}

\normalsize{} %To Recover.

\section{New Command}

It's not recommended to set a font or a size for some texts
directly,you should pack it in a style and apply the style to all
the texts for which you want to set the font and the size.

Use
$\backslash$newcommand\{\emph{name}\}[\emph{num}]\{\emph{definition}\}\
to pack styles or other commands.\\

Use
$\backslash$renewcommand\{\emph{name}\}[\emph{num}]\{\emph{definition}\}\
to repack it.\\

For example:\\

\newcommand{\Ltt}[1]{\Large{\texttt{#1}}}

\Ltt{Large Typewriter}

\renewcommand{\Ltt}[1]{\Huge{\textsf{#1}}}

\Ltt{Huge Sans Serif}

\normalsize{}
\chapter{Useful Environments}
\section{Lists}
\subsection{Itemize}
Default Style:
\begin{itemize}
\item Apple
\item Pear
\item Banana
\end{itemize}
Customized Style\footnote{But it looks stupid.}:
\begin{itemize}
\item[*] Eye
\item[*] Nose
\item[*] Ear
\end{itemize}

\subsection{Enumerate}
\begin{enumerate}
\item Point
\item Line
\item Polygon
\end{enumerate}

\subsection{Description}
I prefer calling it definitions.
\begin{description}
\item[erkl\" aren]German word,meaning ``explain''.
\item[kl\" aren]German word,meaning ``clear''.
\end{description}

\section{Aligning}

\subsection{Flushleft}
\begin{flushleft}
This text is\\ left-aligned. \LaTeX{} is not trying to\\ make each
line the same length.
\end{flushleft}

\subsection{Flushright}
\begin{flushright}
This text is\\ left-aligned. \LaTeX{} is not trying to\\ make each
line the same length.
\end{flushright}

\subsection{Center}
\begin{center}
In the tremendous sea of faces.\\
We met,gathered then seperated.\\
I hope our friendship will go beyond time and space.\\
Wish you
happiness and merriment.
\end{center}

\section{Quoting}
\subsection{Quote}
In \emph{The Winter's Tale},Shakespear said:
\begin{quote}
I should leave grazing,were I of your flock,and only live by gazing.
\end{quote}
\subsection{Quotation}
Quoting paragraphs:
\begin{quotation}
This fertile and sheltered tract of country, in which the fields are
never brown and the springs never dry, is bounded on the south by
the bold chalk ridge\ldots

The district is of historic, no less than of topographical interest.
The Vale was known in former times as the Forest of White Hart, from
a curious legend of King Henry III's reign\ldots

The forests have departed, but some old customs of their shades
remain. Many, however, linger only in a metamorphosed or disguised
form. \ldots
\end{quotation}

\subsection{Verse}
It's used for quoting poems.

\begin{verse}
\begin{center}
I've Got A Pain In My Sawdust

w. Henry Edward Warner m. Herman Avery Wade
\end{center}

\ldots

I've got a pain in my sawdust, \\
That's what's the matter with me;\\
Something is wrong with my little inside,\\
I'm just as sick as can be.

Don't let me faint, \\
someone get me a fan,\\
Someone else run for the medicine man,\\
Ev'ryone hurry as fast as you can,\\
I've got a pain in my sawdust.

\ldots

Oh, sad was the day for the little bisque doll,\\
For they cut all her stitches away,\\
And looked for the seat of the terrible ache;\\
``T'was a delicate task", they all say,

For none of the surgeons had ever before\\
Performed on a dolly's inside,\\
They tried to restuff her but didn't know how,\\
And this was her wail as she died;\\
I've got a pain\ldots
\end{verse}

\section{Just Show It In The Way That It's Typed}
\begin{verbatim}
Use the pair of
                                  \begin{verbatim}
                And
\end{verbatim}
\verb|                  \end{verbatim}|
\begin{verbatim}
                Or
                          \verb| Contents that you want them to be
                          shown in the way it's typed |

      Actually,| pair can be replaced by any symbol pair
                                  like + # @ &
                  expect * and space,

      I guess it's prepared for CODES.
\end{verbatim}

\section{Tabular}

Now it's time to create tables.

\verb|\begin{Tabular}{|\emph{Table Style}\verb|}|

\emph{Table Contents}

\verb|\end{Tabular}|

\subsection{Table Style}
\emph{Table Style}\ is not responsible for the creation of horizonal
lines in the table,that's the responsibility of \emph{Table
Contents}'s.
\begin{description}
\item[l,r,c] creates a row that is left-aligned,right-aligned or
centered.

\item[p\{\emph{width}\}] creates a row by the given width.

\item[$|$] creates a vertical line to separate rows.

\item[@\{\emph{symbol}\}]separate rows with the symbol
\emph{symbol}.
\end{description}

\subsection{Table Contents}
\begin{description}
\item[\&] jump to the next row.

\item[$\backslash\backslash$] jump to the next line.

\item[$\backslash$hline] creates a horizonal line through all rows.

\item[$\backslash$cline\{\emph{i}-\emph{j}\}] creates a horizonal line from row \emph{i} to row \emph{j}.
\end{description}

\subsection{Examples}
An ordinary table:\\

\begin{tabular}{|r|l|}
\hline
7C0 & hexadecimal \\
3700 & octal \\ \cline{2-2}
11111000000 & binary \\
\hline \hline
1984 & decimal \\
\hline
\end{tabular}
\\\\

Using \verb|@{}|\ to coordinate the radix point:\\

\begin{tabular}{c r @{.} l}
Pi expression &
\multicolumn{2}{c}{Value} \\
\hline
$\pi$ & 3&1416 \\
$\pi^{\pi}$ & 36&46 \\
$(\pi^{\pi})^{\pi}$ & 80662&7 \\
\end{tabular}

\section{Where To Put It?Float It!}
\verb|begin{figure}[|\emph{placement specifier}]

or

\verb|begin{table}[|\emph{placement specifier}]
\\\\\\
\begin{tabular}{c @{} c}

placement specifier & where to put it\\

\hline

h & put it on the current page. \\

t & put it on the top of a page.\\

b & put it on the bottom of a page.\\

p & put it on an individual page.\\

! & place it rigidly as placement specifier requested.\\

\hline


\end{tabular}
\\\\

Figure~\ref{Empty} is an example of Pop-Art.
\begin{figure}[!hbp]
\makebox[\textwidth]{\framebox[5cm]{\rule{0pt}{5cm}}} \caption{Five
by Five in Centimetres.} \label{Empty}
\end{figure}

\section{Protect Fragile Commands\protect\footnote{For example,protecting my footnote.}}
Without \verb|\protect|,I can't even put a footnote for the title of
a section.

\chapter{Math Formulas}
Yeah!Eventually we've reached the most powerful function and also
the most exciting part of \LaTeX---Math formulas!We might use the
AMS-\LaTeX\ or other macros.
\section{Math Modes}
\subsection{Math Formulas In Paragraphs}

There are three choice:

\verb|\begin{math}|\emph{Formula}\verb|\end{math}|

\verb|$|\emph{Formula}\verb|$|

\verb|\(| \emph{Formula} \verb|\)|

\subsection{Math Formulas In Display Mode}
The formula will stand alone,and will not be enumerated.

\verb|\begin{displaymath}|\emph{Formula}\verb|\end{displaymath}|

\verb|\[| \emph{Formula} \verb|\]|

\subsection{Math Formulas In Equation Mode}
The formula will stand alone,and will be enumerated.If we use
\verb|\begin{equation*}|,the equations will not be enumerated.

\verb|\begin{equation}|\emph{Formula}\verb|\end{equation}|

\subsection{Examples}
Formulas in paragraph,$\lim_{n \to \infty} \sum_{k=1}^n
\frac{1}{k^2} = \frac{\pi^2}{6}$.

Formulas in display mode:

\begin{displaymath}
\lim_{n \to \infty} \sum_{k=1}^n \frac{1}{k^2} = \frac{\pi^2}{6}
\end{displaymath}

Formulas in equation mode\footnote{With Package \emph{amsfonts}\ or
\emph{amssymb},we can have the blackboard bold font for sets.}:

\begin{equation}
\forall x \in \mathbb{R}: \qquad x^{2} \geq 0
\end{equation}

Every letter in math mode will be treated as a variable,except when
it's in \verb|\textrm{}|\ or \verb|\mathrm{}|:

\begin{equation}
x^{2} \geq 0\qquad \textrm{for all }x\in\mathbb{R}
\end{equation}

The difference between \verb|\mathrm{}|\ and \verb|\textrm{}|\ as
follows\footnote{If Package \emph{amsmath} is used,there will be no
difference.}:

\begin{equation}
2^{\textrm{nd}} \quad 2^{\mathrm{nd}}
\end{equation}

See Section \ref{Align}\ on Page \pageref{Align} to learn to deal
with equations.

\section{Math Spacing}
\subsection{Units}
\begin{tabular}{c @{\quad} l}

pt & point,$\frac{1}{72.27}$\ inch.\\\\

bp & Adobe big point,$\frac{1}{72}$\ inch.\\\\

pc & pica,12pt\\\\

mm & millimeter\\\\

cm & centimeter\\\\

in & inch,25.4mm\\\\

em & similar to the width of M.\\\\

ex & similar to the height of x.

\end{tabular}


\subsection{Spaces}

\begin{tabular}{c @{\quad} l}

\verb|\,| & $\frac{3}{18}$\ quads.\\\\

\verb|\:| & $\frac{4}{18}$\ quads.\\\\

\verb|\;| & $\frac{5}{18}$\ quads.\\\\

\verb|\!| & $-\frac{3}{18}$\ quads.\\\\

\verb|\quad|\ & 1 quad,similar to the width of M.\\\\

\verb|\qquad|\ & 2 quads.

\end{tabular}
\subsection{Phantom}

\verb|\phantom|\ reserves room for something that exists but not to
be displayed.

See examples:\\

\begin{tabular}{c @{\quad} l}

${}^{12}_{6}\textrm{C}$ & \verb|{}^{12}_{6}\textrm{C}|\\\\

${}^{12}_{\phantom{0}6}\textrm{C}$ & \verb|{}^{12}_{\phantom{0}6}\textrm{C}|\\\\

$\Gamma_{ij}^{k}$ & \verb|\Gamma_{ij}^{k}|\\\\

$\Gamma_{ij}^{\phantom{ij}k}$ & \verb|\Gamma_{ij}^{\phantom{ij}k}|

\end{tabular}\\\\

See Section \ref{Sup and Sub} on Page \pageref{Sup and Sub} to learn
\emph{Superscript and Subscript}.

\section{Math Sizing}
\subsection{Setting Size}

\[\displaystyle{\backslash\mathrm{displaystyle\{\}}}\]
\[\textstyle{\backslash\mathrm{textstyle\{\}}}\]
\[\scriptstyle{\backslash\mathrm{scriptstyle\{\}}}\]
\[\scriptscriptstyle{\backslash\mathrm{scriptscriptstyle\{\}}}\]
\subsection{Pairing size}

\begin{tabular}{|l|}

\hline \verb|\Bigg(\bigg(\Big(\big(\big)\Big)\bigg)\Bigg)|\\
\hline
\end{tabular}

\[\Bigg(\bigg(\Big(\big(\big)\Big)\bigg)\Bigg)\]


\begin{tabular}{|l|}

\hline\verb@\Bigg\{\bigg\{\Big\{\big\{\big\}\Big\}\bigg\}\Bigg\}@\\
\hline

\end{tabular}

\[\Bigg\{\bigg\{\Big\{\big\{\big\}\Big\}\bigg\}\Bigg\}\]

\begin{tabular}{|l|}

\hline\verb@\Bigg\|\bigg\|\Big\|\big\|\big\|\Big\|\bigg\|\Bigg\|@\\
\hline

\end{tabular}

\[\Bigg\|\bigg\|\Big\|\big\|\big\|\Big\|\bigg\|\Bigg\|\]

\begin{tabular}{|l|}

\hline\verb|1 + ( \frac{1}{ 1-x^{2} } ) ^3|\\
\hline

\end{tabular}

\[1 + ( \frac{1}{ 1-x^{2} } ) ^3\]

Use the \verb|\left| and \verb|\right| pairs to determine the
correct sizes of symbols.
\\

\begin{tabular}{|l|}

\hline\verb|1 + \left( \frac{1}{ 1-x^{2} } \right) ^3|\\
\hline

\end{tabular}

\[1 + \left( \frac{1}{ 1-x^{2} } \right) ^3\]

If there is nothing on the right side,use ``\verb|\right.|''.
\\

\begin{tabular}{|l|}
\hline

\verb|y = \left\{| \\
\verb|           \begin{array}{ l l }|\\
\verb|                a   &  x \leq -5\\\\|\\
\verb|                b+x &  -5 < x < 7\\\\|\\
\verb|                l   &  x \geq 7|\\
\verb|           \end{array}|\\
\verb|    \right.|\\
\hline
\end{tabular}\\

\[y = \left\{ \begin{array}{ll}
a & x \leq -5\\\\
b+x & -5 < x < 7\\\\\
l & x \geq 7
\end{array} \right.\]\\

See Section \ref{Array} on Page \pageref{Array} to learn Environment
\emph{Array}.


\subsection{Bold Fonts}
\begin{tabular}{c @{\quad} c}

\verb|$\mu, M$| & $\mu, M$\\\\

\verb|$\mathbf{\mu},\mathbf{M}$| & $\mathbf{\mu},\mathbf{M}$\\\\

\verb|\mbox{\boldmath $\mu, M$}| & \mbox{\boldmath $\mu, M$}

\end{tabular}\\\\

\verb|\boldmath| must be used outside the math mode,or in the
\verb|\mbox{}| in the math mode.
\section{Math Fonts}
\begin{tabular}{l @{\quad} c }

\verb|\mathrm{ABCdef}| & $\mathrm{ABCdef}$\\\\

\verb|\mathit{ABCdef}| & $\mathit{ABCdef}$\\\\

\verb|\mathnormal{ABCdef}| & $\mathnormal{ABCdef}$\\\\

\verb|\mathcal{ABC}| & $\mathcal{ABC}$\\\\

\verb|\mathfrak{ABCdef}| & $\mathfrak{ABCdef}$\\\\

\verb|\mathbb{ABC}| & $\mathbb{ABC}$\\\\

\verb|\mathtt{ABCdef}| & $\mathtt{ABCdef}$\\\\

\verb|\mathsf{ABCdef}| & $\mathsf{ABCdef}$\\\\

\verb|\mathbf{ABCdef}| & $\mathbf{ABCdef}$
\end{tabular}
\\\\

The command \verb|\mathfrak{ABCdef}|\ requires Package
\emph{eufrak}.
\section{Frequently Used Symbols}
\subsection{Dots}
Observe carefully,then you will see that \verb|\ldots| generates lower dots than \verb|\cdots|.\\

\begin{tabular}{l @{\quad} c}

\verb|\ldots| & $\ldots$\\\\

\verb|\cdot| & $\cdot$\\\\

\verb|\cdots| & $\cdots$\\\\

\verb|\vdots| & $\vdots$\\\\

\verb|\ddots| & $\ddots$

\end{tabular}\\\\

See a practical example in Section \ref{Array} on Page
\pageref{Array}.

\subsection{Superscript and Subscript}\label{Sup and Sub}
\begin{tabular}{l @{\quad} c}

\verb|a_1| & $a_1$\\\\

\verb|x^2| & $x^2$\\\\

\verb|a_{ij}| & $a_{ij}$\\\\

\verb|x^{y^z}| & $x^{y^z}$\\\\

\verb|e^{x^2} \neq {e^x}^2| & $e^{x^2} \neq {e^x}^2$\\\\

\verb|{}^{12}_{\phantom{0}6}\mathrm{C}| &
${}^{12}_{\phantom{0}6}\mathrm{C}$

\end{tabular}
\\

\^\ or \_\ might tinily changes its position and meaning,for
example,see Section \ref{+*} on Page \pageref{+*},they turn out to
be upper and lower limits.

\subsection{Square Root}
\begin{tabular}{l @{\quad} c}

\verb|\sqrt{a}| & $\sqrt{a}$\\\\

\verb|\sqrt[n]{a}| & $\sqrt[n]{a}$\\\\

\verb|\surd| & $\surd$\\\\

\verb|\sqrt{ x^{2}+\sqrt{y} }| & $\sqrt{ x^{2}+\sqrt{y} }$

\end{tabular}

\subsection{Line and Brace}

\begin{tabular}{l @{\quad} c}

\verb|\overline{m+n}| & $\overline{m+n}$\\\\

\verb|\underline{m+n}| & $\underline{m+n}$\\\\

\verb|\underbrace{ a+b+\cdots+z }_{26}| & $\underbrace{ a+b+\cdots+z
}_{26}$\\\\

\verb|\overbrace{ a+b+\cdots+z }^{26}| & $\overbrace{ a+b+\cdots+z }^{26}$\\\\

\end{tabular}

\subsection{Vector}

\begin{tabular}{l @{\quad} c}

\verb|\vec a| & $\vec a$\\\\

\verb|\overrightarrow{AB}| & $\overrightarrow{AB}$\\\\

\verb|\overleftarrow{AB}| & $\overleftarrow{AB}$

\end{tabular}

\subsection{Fraction}
\begin{displaymath}
\begin{array}{l c}
\verb|1/2| & 1/2\\\\
\verb|3\frac{1}{2}| & 3\frac{1}{2}\\\\
\end{array}
\end{displaymath}\\

\begin{tabular}{|l|}
\hline

\verb|y^{}_{\mathrm{A}}= \displaystyle\frac| \\
\verb|{|\\
\verb|    p^*_{\mathrm{A}} x^{}_{\mathrm{A}}| \\
\verb|}| \\
\verb|{| \\
\verb|    p^*_{\mathrm{A}} x^{}_{\mathrm{A}}| \\
\verb|+| \\
\verb|    p^*_{\mathrm{B}}(1-x^{}_{\mathrm{A}})| \\
\verb|}| \\
\hline
\end{tabular}\\

\begin{displaymath}
y^{}_{\mathrm{A}}= \displaystyle\frac
{
    p^*_{\mathrm{A}} x^{}_{\mathrm{A}}
}
{
    p^*_{\mathrm{A}} x^{}_{\mathrm{A}}
+
    p^*_{\mathrm{B}}(1-x^{}_{\mathrm{A}})
}
\end{displaymath}\\

\begin{tabular}{|l| }
\cline{1-1}
\verb|N_{\mathrm{OG}}= \frac| \\
\verb|{| \\
\verb|    y^{}_1 - y^{}_2| \\
\verb|}| \\
\verb|{| \\
\verb|    \displaystyle\frac| \\
\verb|    {| \\
\verb|        (y^{}_1 - y^*_1) - (y^{}_2 - y^*_2)| \\
\verb|    }| \\
\verb|    {| \\
\verb|        \ln\frac{y^{}_1 - y^*_1}{y^{}_2 - y^*_2}| \\
\verb|    }| \\
\verb|}| \\
\cline{1-1}
\end{tabular}
\\

\begin{displaymath}
N_{\mathrm{OG}}= \frac
{
    y^{}_1 - y^{}_2
}
{
    \displaystyle\frac
    {
        (y^{}_1 - y^*_1) - (y^{}_2 - y^*_2)
    }
    {
        \ln\frac{y^{}_1 - y^*_1}{y^{}_2 - y^*_2}
    }
}
\end{displaymath}
\subsection{Binomial Coefficients And Customized Fraction}
Without Package \emph{amsmath},we can only use

\verb|{n \choose m}|\ or\ \verb|{x \atop y+2}| to generate the
binomial coefficients or similar structures:
\begin{displaymath}
{n \choose m} \qquad {x \atop y+2}
\end{displaymath}

With Package \emph{amsmath},we can use \verb|\binom{n}{m}|\ to
generate binomial coefficients:

\begin{displaymath}
\binom{n}{m}
\end{displaymath}

But the most powerful part is the command
\verb|\genfrac{}{}{}{}{}{}|,it has six
arguments:\\

Argument 5 and 6 are the numerator and the denominator.

Argument 1 and 2 are the left delimiter and the right delimiter.`.'\
means there are no delimiter.

Argument 3 is the thickness of the line between the numerator and
the denominator,set it to 0pt to make it invisible.

Argument 4 is the size of the numerator and the
denominator.displaystyle = 0, textstyle = 1, scriptstyle = 2,
scriptscriptstyle = 3.\\

The Command \verb|\genfrac{(}{)}{0pt}{}{n}{m}|\ works exactly the
same as \verb|\binom{n}{m}|.
\begin{displaymath}
\genfrac{(}{)}{0pt}{}{n}{m}
\end{displaymath}

\subsection{Sum,Product And Calculus}\label{+*}


\verb|\sum_{i=1}^{n}|:

\[\sum_{i=1}^{n}\]

\verb|\int_{0}^{\frac{\pi}{2}}|:

\[\int_{0}^{\frac{\pi}{2}}\]

\verb|\prod_\epsilon|:

\[\prod_\epsilon\]


\verb|\iint_D|:

\[\iint_D\]

\verb|\iiint_V|:

\[\iiint_V\]

\verb|\idotsint_{\mathbb{R^{\mathrm n}}}|:

\[\idotsint_{\mathbb{R^{\mathrm n}}}\]

\section{Math Array}
\subsection{Array}\label{Array}
\begin{tabular}{|l|}
\hline
\verb|\mathbf{X} = \left( \begin{array}{ccc}|\\
\verb|x_{11} & x_{12} & \ldots \\|\\
\verb|x_{21} & x_{22} & \ldots \\|\\
\verb|\vdots & \vdots & \ddots|\\
\hline
\end{tabular}\\\\

\begin{displaymath}
\mathbf{X} = \left( \begin{array}{ccc}
x_{11} & x_{12} & \ldots \\
x_{21} & x_{22} & \ldots \\
\vdots & \vdots & \ddots
\end{array} \right)
\end{displaymath}\\


\begin{tabular}{|l|}
\hline
\verb@\left(\begin{array}{c|c}@\\
\verb|1 & 2 \\|\\
\verb|\hline|\\
\verb|3 & 4|\\
\verb|\end{array}\right)|\\
\hline
\end{tabular}\\

\begin{displaymath}
\left(\begin{array}{c|c}
1 & 2 \\
\hline 3 & 4
\end{array}\right)
\end{displaymath}

\subsection{Eqnarray}

Environment \emph{Eqnarray} must be used outside the math
mode,because it's an environment similar to Environment
\emph{Equation}.\\

\begin{tabular}{|l|}
\hline
\verb|\begin{eqnarray}|\\
\verb|f(x) & = & \cos x \\|\\
\verb|f'(x) & = & -\sin x \\|\\
\verb|\int_{0}^{x} f(y)\mathrm dy & = & \sin x|\\
\verb|\end{eqnarray}|\\
\hline
\end{tabular}\\


\begin{eqnarray}
f(x) & = & \cos x \\
f'(x) & = & -\sin x \\
\int_{0}^{x} f(y)\mathrm dy & = & \sin x
\end{eqnarray}

Put the whole environment in \verb|{\setlength\arraycolsep{2pt}  }|,
then the space around ``='' will be smaller:\\

{\setlength\arraycolsep{2pt}
\begin{eqnarray}
f(x) & = & \cos x \\
f'(x) & = & -\sin x \\
\int_{0}^{x} f(y)\mathrm dy & = & \sin x
\end{eqnarray}
}

Use Environment \emph{Eqnarray} to split a long equation:\\

\begin{tabular}{|l|}
\hline
\verb|\begin{eqnarray}|\\
\verb|\lefteqn{ \cos x = 1 -\frac{x^{2}}{2!} +{} }|\\
\verb|\nonumber\\|\\
\verb|& & {}+\frac{x^{4}}{4!} -\frac{x^{6}}{6!}+{}\cdots|\\
\verb|\end{eqnarray}|\\
\hline
\end{tabular}\\

\begin{eqnarray}
\lefteqn{ \cos x = 1 -\frac{x^{2}}{2!} +{} }
\nonumber\\
& & {}+\frac{x^{4}}{4!} -\frac{x^{6}}{6!}+{}\cdots
\end{eqnarray}

\subsection{Align}\label{Align}
With Package \emph{amsmath},we can use Environment \emph{Align}\ to
deal with equations.Use \& to tell \LaTeX\ how to align.Observe the
usage of Environment \emph{Subequations}.

\begin{tabular}{|l|}
\hline
\verb|\begin{subequations}|\\
\verb|\begin{align}|\\
\verb|x & \equiv 2 \pmod 3 \\|\\
\verb|x & \equiv 3 \pmod 5 \\|\\
\verb|\end{align}|\\
\verb|\end{subequations}|\\
\hline
\end{tabular}\\

\begin{subequations}
\begin{align}
x & \equiv 2 \pmod 3 \\
x & \equiv 3 \pmod 5 \\
x & \equiv 2 \pmod 7
\end{align}
\end{subequations}

\section{Math Theorem}
To initialize a Theorem System,we should put the following
declaration in the preamble area(The area between
\verb|\documentclass| and \verb|\begin{document}| is called the
\emph{preamble}).\\

\begin{tabular}{|l|}
\hline
\verb|\newtheorem{|\emph{name}\verb|}[|\emph{counter}\verb|]{|\emph{text}\verb|}[|\emph{section}\verb|]|\\
\hline
\end{tabular}\\

\emph{name}\ is the identifier of the theorem for \LaTeX,and
\emph{text}\ will be displayed as the name of the theorem.

\emph{counter}\ and \emph{section}\ shall not be given at the same
time.\emph{counter}\ is the identifier of the counter for \LaTeX,it
shall be the name of another theorem.If we replace \emph{section}\
by ``section'',``chapter'' or ``subsection'',the theorem will be
enumerated by section,chapter or subsection.\\

For example,declare like this in the preamble,\\

\begin{tabular}{|l|}
\hline \verb|\newtheorem{Theorem}{定理}[section]|\\
\hline
\end{tabular}\\

and then,\\

\begin{tabular}{|l|}
\hline
\verb|\begin{Theorem}|\\
\verb|sth.|\\
\verb|\end{Theorem}|\\
\hline
\end{tabular}\\

will generate:\\

\begin{Theorem}
sth.
\end{Theorem}

And declare like this in the preamble,\\

\begin{tabular}{|l|}
\hline \verb|\newtheorem{FirstTheorem}{定理}|\\
\hline \verb|\newtheorem{NextTheorem}[Theorem]{定理}|\\
\hline
\end{tabular}\\

and then,\\

\begin{tabular}{|l|}
\hline
\verb|\begin{FirstTheorem}|\\
\verb|Something.|\\
\verb|\end{FirstTheorem}|\\
\verb|\begin{NextTheorem}|\\
\verb|Some other thing.|\\
\verb|\end{NextTheorem}|\\
\hline
\end{tabular}\\

\begin{FirstTheorem}
Something.
\end{FirstTheorem}

\begin{NextTheorem}
Some other thing.
\end{NextTheorem}

\section{Symbol Lists}
\newcommand{\qqq}{\qquad&}
See its source to know how to type it.If a symbol is with a
superscript ``A'' on its left side,then it's only provided by
\emph{amsmath}.

\begin{displaymath}
\begin{aligned}
& 0 \qqq 1 \qqq 2 \qqq 3 \qqq 4 \qqq 5 \qqq 6 \qqq 7 \qqq 8\\
& 1 \qqq \bar{a} \qqq \acute{a} \qqq \check{a} \qqq \grave{a} \qqq
\dot{a} \qqq \ddot{a} \qqq \hat{a} \qqq \widehat{A}\\
& 2 \qqq \vec{a} \qqq \breve{a} \qqq \tilde{a} \qqq \widetilde{A}
\qqq \\
& 3 \qqq \alpha \qqq \beta \qqq \gamma \qqq \delta \qqq
\epsilon \qqq \varepsilon \qqq \zeta \qqq \eta\\
& 4 \qqq \theta \qqq \vartheta \qqq \iota \qqq \kappa \qqq
\lambda \qqq \mu \qqq \nu \qqq \xi \\
& 5  \qqq o \qqq \pi \qqq \varpi \qqq \rho \qqq \varrho \qqq \sigma
\qqq \varsigma \qqq \tau\\
& 6 \qqq \upsilon \qqq \phi \qqq \varphi \qqq \chi \qqq \psi \qqq
\omega \\
& 7 \qqq \Gamma \qqq \Delta \qqq \Theta \qqq \Lambda \qqq \Xi \qqq
\Pi \qqq \Sigma \qqq \Upsilon \\
& 8 \qqq \Phi \qqq \Psi
\end{aligned}
\end{displaymath}

\begin{displaymath}
\begin{aligned}
& 0 \qqq 1 \qqq 2 \qqq 3 \qqq 4 \qqq 5 \qqq 6 \qqq 7 \qqq 8\\
& 1 \qqq < \qqq \le \qqq > \qqq \ge \qqq = \qqq \equiv \qqq \ll
\qqq \gg\\
& 2 \qqq \prec \qqq \preceq \qqq \succ \qqq \succeq \qqq \sim \qqq
\simeq \qqq \approx \qqq \cong\\
& 3 \qqq \subset \qqq \subseteq \qqq \supset \qqq \supseteq \qqq
{}^A\sqsubset \qqq \sqsubseteq \qqq {}^A\sqsupset \qqq \sqsupseteq\\
& 4 \qqq \doteq \qqq \propto \qqq {}^A\Join \qqq \bowtie \qqq \vdash
\qqq \dashv \qqq \perp \qqq \models\\
& 5 \qqq \mid \qqq \parallel \qqq \smile \qqq \frown \qqq \asymp
\qqq : \qqq \notin \qqq \neq\\
& 6 \qqq + \qqq - \qqq \pm \qqq \mp \qqq \cdot \qqq \times \qqq /
\qqq \div \qqq \\
& 7 \qqq \oplus \qqq \ominus \qqq \odot \qqq \otimes \qqq \oslash
\qqq \setminus \qqq \lor \qqq \land\\
& 8 \qqq \cup \qqq \cap \qqq \sqcup \qqq \sqcap \qqq \bigtriangleup
\qqq \bigtriangledown \qqq \triangleleft \qqq \triangleright
\end{aligned}
\end{displaymath}

\begin{displaymath}
\begin{aligned}
& 0 \qqq 1 \qqq 2 \qqq 3 \qqq 4 \qqq 5 \qqq 6 \qqq 7 \qqq 8\\
& 1 \qqq {}^A\lhd \qqq {}^A\rhd \qqq {}^A\unlhd \qqq {}^A\unrhd \qqq
\star \qqq \ast \qqq \circ \qqq \bigcirc\\
& 2 \qqq \bullet \qqq \diamond \qqq \uplus \qqq \amalg \qqq \dagger
\qqq \ddagger \qqq \wr \qqq {}\\
& 3 \qqq \sum \qqq \prod \qqq \coprod \qqq \bigsqcup \qqq \bigcup
\qqq \bigcap \qqq \int \qqq \oint\\
& 4 \qqq \bigvee \qqq \bigwedge \qqq \bigoplus \qqq \bigotimes \qqq
\bigodot \qqq \biguplus\\
& 5 \qqq \gets \qqq \to \qqq \longleftarrow \qqq \longrightarrow
\qqq \Leftarrow \qqq \Rightarrow \qqq \Longleftarrow \qqq
\Longrightarrow\\
& 6 \qqq \uparrow \qqq \downarrow \qqq \Uparrow \qqq \Downarrow \qqq
\leftrightarrow \qqq \Leftrightarrow \qqq \longleftrightarrow \qqq
\Longleftrightarrow\\
& 7 \qqq \updownarrow \qqq \Updownarrow \qqq \nearrow \qqq \searrow
\qqq \swarrow \qqq \nwarrow \qqq \leadsto \qqq \iff\\
& 8 \qqq \mapsto \qqq \longmapsto \qqq \hookleftarrow \qqq
\hookrightarrow \qqq \leftharpoonup \qqq \rightharpoonup \qqq
\leftharpoondown \qqq \rightharpoondown\\
& 9 \qqq \rightleftharpoons
\end{aligned}
\end{displaymath}


\begin{displaymath}
\begin{aligned}
& 0 \qqq 1 \qqq 2 \qqq 3 \qqq 4 \qqq 5 \qqq 6 \qqq 7 \qqq 8\\
& 1 \qqq ( \qqq ) \qqq \lbrack \qqq \rbrack \qqq \lbrace \qqq
\rbrace \qqq \langle \qqq \rangle\\
& 2 \qqq \lfloor \qqq \rfloor \qqq \lceil \qqq \rceil \qqq \vert
\qqq | \qqq \Vert \qqq \|\\
& 3 \qqq \lgroup \qqq \rgroup \qqq \lmoustache \qqq \rmoustache \qqq
\arrowvert \qqq \Arrowvert \qqq \bracevert
\end{aligned}
\end{displaymath}

To be continued...

\end{document}
